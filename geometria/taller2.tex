\documentclass[12pt]{article}

\usepackage[spanish]{babel}
\usepackage[utf8]{inputenc}
\usepackage{latexsym,amsfonts,amssymb,amsthm,amsmath}
\usepackage{amsmath}
\setlength{\parindent}{0in}
\setlength{\oddsidemargin}{0in}
\setlength{\textwidth}{6.5in}
\setlength{\textheight}{8.8in}
\setlength{\topmargin}{0in}
\setlength{\headheight}{18pt}



\title{Taller de Geometr\'ia Diferencial semana 4 }
\author{Juan Sebasti\'an Gait\'an}

\begin{document}

\maketitle

\vspace{0.5in}



\subsection*{Ejercicio 3.1}
Para $r,a,b\in \mathbb{R}^{+}$  considere las helices parametrizadas en $\mathbb{R}^{3}$ por:
\begin{equation*}
\begin{split}
\gamma_{1}&: r \mapsto (r \cdot \cos (at), r \cdot \sin (at), b \cdot (at)), \\
\gamma_{2}&: r \mapsto (r \cdot \cos (-at), r \cdot \sin (-at) , b \cdot (at)).
\end{split}
\end{equation*}
Calcule sus curvaturas $\kappa_1 , \kappa_2 $ y las torsiones $\tau_1 , \tau_2$ respectivamente. Encuentre un movimiento euclidiano $\Phi: \mathbb{R}^3 \rightarrow \mathbb{R}^3$ tal que $\gamma_2=\Phi \circ \gamma_1$. ?`$\Phi$ conserva orientaci\'on?
\begin{proof}
\end{proof}

%Leave space for comments!


\subsection*{Ejercicio 3.3}
 Demuestre que la curva $\gamma :(-\frac{\pi}{2},\frac{\pi}{2}) \rightarrow \mathbb{R}^3$ con
\begin{equation*}
   \gamma :t \mapsto (2 \cos^2 t -3 , \sin t - 8 , 3 \sin ^2 t + 4)
\end{equation*}
es regular. Determine si la imagen de $\gamma$ est\'a contenida en:
\begin{enumerate}
  \item una linea recta en $\mathbb{R}^3$ o no.
  \item un plano en $\mathbb{R}^3$ o no.
\end{enumerate}


\begin{proof}
Note que las curvas parametrizadas por $\gamma_{1}$ y  $\gamma_{2}$ son c\'irculos y por lo tanto las curvaturas son $\frac{1}{r}$ y $\frac{-1}{r}$ respectivamente. Adem\'as, el movimiento r\'igido est\'a dado por:
\begin{equation*}
\begin{pmatrix} 1 & 0 \\ 0 & -1 \end{pmatrix}.
\end{equation*}
\end{proof}

%Leave more space for comments!
\subsection*{Ejercicio 3.5}
Sea $\gamma : I \rightarrow \mathbb{R}^3$ un a $C^2$-curva regular en $\mathbb{R}^3$ con curvatura no nula. Entonces la torsion satisface:
\begin{equation*}
\tau (t)=\frac{\det [\gamma '(t), \gamma ''(t), \gamma ''' (t)]}{|\gamma'(t) \times \gamma '' (t)|^2 }
\end{equation*}
\begin{proof}
Tome $\alpha= \gamma \circ \tau$ una reparametrizaci\'on de $\gamma$ con $\alpha: (0,L(\gamma)) \rightarrow \mathbb{R}^2$ por longitud de arco, $\gamma:I \rightarrow \mathbb{R}^2$ y $\tau: (0,L(\gamma)) \rightarrow (a,b)$ como hab\'iamos probado para curvas con parametrizaci\'on natural que $\kappa(s)_{\alpha}=<\dot{T}(s),N(s)>$ y tome $\gamma(t)=(x(t),y(t))$.
\begin{equation*}
\begin{split}
T&=\dot{\alpha}\\
T&=(\gamma \circ \tau)'(s)\\
T&=\gamma'(\tau(s) \cdot \tau'(s))\\
T&=(x'(\tau(s)), y'(\tau(s))\tau'(s))\\
T(s)&=(\tau'(s)x'(\tau(s))), (\tau'(s)y'(\tau(s)))\\
\dot{T}(s)&=[\tau''(s)x'(\tau(s)) + \tau^2(s)x''(\tau(s))\tau'(s), \tau''(s)y'(\tau(s))+\tau'(s)y''(\tau(s))\tau'(s)] \\
&=[\tau''(s)x'(\tau(s))+(\tau'(s))^2x''(\tau(s)), \tau''(s)y'(\tau(s))+(\tau'(s))^2y''(\tau(s))]
\end{split}
\end{equation*}

Adem\'as, note que:
\begin{equation*}
N(s)=\begin{pmatrix} 0 & -1 \\ 1 & 0 \end{pmatrix}\begin{pmatrix} x'(\tau(s))\cdot \tau'(s)\\ y'(\tau(s))\cdot \tau'(s)\end{pmatrix}=\begin{pmatrix} -y'(\tau(s))\cdot \tau'(s)\\ x'(\tau(s))\cdot \tau'(s)\end{pmatrix}.
\end{equation*}
Y por lo tanto, se tiene que:
\begin{equation*}
\begin{split}
\kappa (s)&=\langle \dot{T}(s),N(s) \rangle\\
&=-y'(\tau(s))x'(\tau(s))\tau'(s)\tau''(s)-y'(\tau(s))\tau'(s)^3x''(\tau(s))+\\
&y'(\tau(s))x'(\tau(s))\tau'(s)\tau''(s)+x'(\tau(s))\tau'(s)^3y''(\tau(s))\\
&=[\tau'(s)]^3[x'(\tau(s))y''(\tau(s))-x'(\tau(s))x''(\tau(s))]
\end{split}
\end{equation*}
y como $\tau'(s)=\frac{1}{\sigma'(\tau(s))}=\frac{1}{|\gamma'(t)|}$, entonces:
\begin{equation*}
\begin{split}
\kappa(s)&=\frac{1}{|\gamma'(t)|^3}[x'(\tau(s))y''(\tau(s))-x'(\tau(s))x''(\tau(s))]\\
&=\frac{1}{|\gamma'(t)|^3}\det\begin{pmatrix} x'(\tau(s)) & x''(\tau(s))\\ y'(\tau(s)) & y''(\tau(s))\end{pmatrix}\\
&=\frac{1}{|\gamma'(t)|^3}\det\begin{pmatrix} x'(t) & x''(t)\\y'(t) & y''(t)\end{pmatrix}
\end{split}
\end{equation*}

\end{proof}

\subsection*{Ejercicio 3.7}
Sea $\gamma : \mathbb R \rightarrow \mathbb R^3$ una $C^2$-curva regular cerrada en $\mathbb R ^3$ con parametrizaci\'on natural. busque una demostraci\'on del teorema de Fenchel:
\begin{equation*}
L(\dot{\gamma})\int_{0}^{P} \kappa(s)ds \geq 2\pi
\end{equation*}
\begin{proof}
Sea $\gamma : [0, l] \rightarrow \mathbb R ^3$ un a curva regular parametrizada por longitud de arco. Como $\gamma$ tiene velocidad unitaria, su funci\'on velocidad $\textbf{v}$, es una curva en $S^2$. \\
Vamos a usar el hecho de que $\gamma $ es cerrada para demostrar que la curva geom\'etrica que describe $\textbf{v}$ interseca a todo circulo maximal de $S^2$. Para esto, sea $P \subseteq \mathbb R ^3$ un subespacio bidimensional arbitrario tal que $G=P \cap S^2$ es un ciculo maximal. Denote por $\textbf{n}$ el vector normal de $P$. Note que un punto en $S^2$ est\'a en $G$ si y solo si es ortogonal a $\textbf{n}$. Como $\frac{d}{dt}\langle \gamma(t), \textbf{n} \rangle=\langle \textbf{v} (t),\textbf{n}\rangle $, y por el teorema fundamental del calculo, se tiene:\\
 $$ \int_{0}^{l}{\langle \textbf{v}(t),\textbf{n} \rangle}dt = \langle \gamma (l),\textbf{n}\rangle -\langle \gamma (0), \textbf{n} \rangle = 0   $$.
 Como el valor medio de $\langle \textbf{v}(t), \textbf{n} \rangle$, entonces, se tiene $\langle \textbf{v}(t_0), \textbf{n} \rangle$ para alg\'un $t_0 \in [0,l]$ y por lo tanto, $\textbf{v}$ interseca todo circulo maximal. Por el lema 2.17, se tiene que la longitud de arco de $\textbf{v}$ es mayor que $2\pi$, de dinde se tiene:
 $$ \int_{0}^{l}{\kappa (t)}dt= \int_{0}^{l}{|\textbf{v}'(t)|}dt\geq \pi.$$
\end{proof}

\end{document}
