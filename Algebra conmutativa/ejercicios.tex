\documentclass{article}
\usepackage[utf8]{inputenc}
\usepackage[spanish]{babel}

\title{ \'Algebra Conmutativa \\ Taller 1}
\author{Juan Sebasti\'an Gait\'an }
\date{July 2018}

\usepackage{amsfonts}
\usepackage{amsthm}
\usepackage{amsmath}
\usepackage{natbib}
\usepackage{graphicx}
\usepackage{geometry}
 \geometry{
 letterpaper,
 total={170mm,257mm},
 left=20mm,
 top=20mm,
 }

\begin{document}

\maketitle
\subsection*{Ejercicio 1}
Sea $A$ un anillo. Muestre que el teorema del binomio es valido en $A$. Esto es, dados $x,y\in A$, y $n\in \mathbb{N}$ se tiene:
\begin{equation}
  (x+y)^n = \sum_{k=0}^{n}{\binom{n}{k}(x^{n-k}y^{k})}
\end{equation}



\end{document}
